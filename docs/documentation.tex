\documentclass[12pt,a4paper]{article}
\usepackage[utf8]{inputenc}
\usepackage[russian]{babel}
\usepackage{geometry}
\usepackage{graphicx}
\usepackage{hyperref}
\usepackage{listings}
\usepackage{xcolor}
\usepackage{amsmath}
\usepackage{enumitem}
\usepackage{tcolorbox}
\usepackage{float}

\geometry{left=2cm,right=2cm,top=2cm,bottom=2cm}

\definecolor{codegreen}{rgb}{0,0.6,0}
\definecolor{codegray}{rgb}{0.5,0.5,0.5}
\definecolor{codepurple}{rgb}{0.58,0,0.82}
\definecolor{backcolour}{rgb}{0.95,0.95,0.92}

\lstset{inputencoding=utf8x, extendedchars=\true}

\lstdefinestyle{mystyle}{
    backgroundcolor=\color{backcolour},   
    commentstyle=\color{codegreen},
    keywordstyle=\color{magenta},
    numberstyle=\tiny\color{codegray},
    stringstyle=\color{codepurple},
    basicstyle=\ttfamily\footnotesize,
    breakatwhitespace=false,         
    breaklines=true,                 
    captionpos=b,                    
    keepspaces=true,                 
    numbers=left,                    
    numbersep=5pt,                  
    showspaces=false,                
    showstringspaces=false,
    showtabs=false,                  
    tabsize=2
}

\lstset{style=mystyle}

\usepackage{dirtree}
\usepackage{forest}

\title{Документация к проекту\\ Secure Bank System}
\author{Бойко Руслан и Осипов Илья}
\date{\today}

\graphicspath{ {./pic/} } 

\begin{document}

\maketitle

\begin{abstract}
Документация описывает архитектуру, функциональные и нефункциональные требования, а также краткое пользовательское руководство для Secure Bank System -- многопользовательской банковской системы с системой безопасности, верификации и шифрования операций.
\end{abstract}

\tableofcontents
\newpage

\section{Функциональные требования}

\subsection{Управление клиентами}
Система обеспечивает полный цикл управления клиентскими учетными записями:
\begin{itemize}
\item Регистрация новых клиентов с обязательной проверкой паспортных данных на уникальность;
\item Аутентификация пользователей по уникальному идентификатору учетной записи и паролю;
\item Процедура верификации клиентов уполномоченными сотрудниками службы безопасности
\end{itemize}

\subsection{Управление банковскими счетами}
Реализована система управления счетами различных категорий:
\begin{itemize}
\item Создание и ведение счетов следующих типов:
  \begin{itemize}
  \item Сберегательные счета (Savings)
  \item Расчётные счета (Checking) 
  \item Кредитные счета (Credit)
  \item Депозитные счета (Deposit)
  \end{itemize}
\item Предоставление информации о текущем балансе и полной истории транзакций по каждому счёту
\item Установка и управление кредитными лимитами для счетов кредитного типа
\end{itemize}

\subsection{Банковские операции}
Система поддерживает проведение стандартных банковских операций:
\begin{itemize}
\item Операции пополнения счетов (deposit)
\item Операции снятия денежных средств (withdraw) с автоматической проверкой доступных лимитов
\item Переводы денежных средств между счетами (transfer) в рамках клиентов системы;
\item Механизм одобрения/отклонения крупных операций службой безопасности банка.
\end{itemize}

\subsection{Система безопасности}
Реализованы многоуровневые механизмы защиты:
\begin{itemize}
\item Шифрование всех данных, хранимых в базе данных системы;
\item Двухуровневая система аутентификации (клиенты банка / сотрудники банка);
\item Система очередей для одобрения операций, превышающих установленные лимиты;
\item Специальные ограничения для новых и неверифицированных пользователей.
\end{itemize}

\subsection{Административные функции}
Для сотрудников банка предусмотрены административные функции:
\begin{itemize}
\item Просмотр и обработка операций, ожидающих одобрения;
\item Верификация новых клиентов банка;
\item Управление процентными ставками по кредитным и депозитным продуктам (в процессе реализации);
\end{itemize}

\section{Системные требования}

\subsection{Требования к производительности}
\begin{itemize}
\item Время отклика системы не должно превышать 100 мс для стандартных банковских операций зачисления, перевода и снятия;
\item Система должна поддерживать несколько одновременных клиентских подключений (и параллельную обработку запросов);
\item Эффективное использование оперативной памяти и ресурсов процессора.
\end{itemize}

\subsection{Требования к безопасности}
\begin{itemize}
\item Обязательное шифрование всех конфиденциальных данных при хранении
\item Применение алгоритмов хеширования для хранения паролей пользователей
\item Комплексная валидация всех входных данных и параметров операций
\item Реализация механизма управления сессиями с автоматическим завершением по таймауту
\end{itemize}

\subsection{Требования к надёжности}
\begin{itemize}
\item Автоматическое сохранение состояния системы при каждом изменении данных;
\item Возможность восстановления работоспособности после аварийных сбоев;
\item Регулярное резервное копирование базы данных с возможностью восстановления.
\end{itemize}

\subsection{Требования к масштабируемости}
\begin{itemize}
\item Модульная архитектура системы, позволяющая добавлять новые функциональные компоненты
\item Четкое разделение клиентской и серверной частей системы
\item Поддержка многопоточности для обработки параллельных запросов
\end{itemize}

\subsection{Требования к совместимости}
\begin{itemize}
\item Кроссплатформенная работа на операционных системах семейства macOS (Linux и Windows в процессе отладки);
\item Использование стандартных сетевых протоколов TCP/IP для клиент-серверного взаимодействия;
\item Независимость от внешних систем управления базами данных.
\end{itemize}

\subsection{Технические требования}
Тестирование программы производилось на следующем сетапе:
\begin{itemize}
\item Язык программирования: C++17 или выше
\item Компилятор: GCC 9.0+, Clang 10.0+, CMake 4.2+ или MSVC 2019+
\item Минимальный объем оперативной памяти: 8 ГБ 
\item Минимальное дисковое пространство: 128 ГБ для установки и работы
\item Сетевой интерфейс для клиент-серверного взаимодействия
\end{itemize}





\newpage
\section{Архитектурная документация}

Secure Bank System реализована по многоуровневой клиент-серверной архитектуре с четким разделением ответственности между компонентами. Архитектура следует принципам модульности и инкапсуляции, что обеспечивает высокую сопровождаемость и масштабируемость системы.

Ключевые архитектурные решения:
\begin{itemize}
    \item Трехуровневая архитектура: клиентский уровень, серверный уровень, уровень данных;
    \item Событийно-ориентированная обработка: асинхронная обработка клиентских подключений;
    \item Многопоточность: изолированные потоки для каждого клиентского соединения;
    \item Пассивный объект базы данных: инкапсуляция логики хранения и сериализации.
\end{itemize}

Дерево проекта: 
\dirtree{%
.1 .
.2 CMakeLists.txt.
.2 Makefile.
% .2 build.
% .3 Makefile.
% .3 data.
% .3 test\_data.
.2 data.
.3 accounts.dat.
.3 accounts.dat.settings.
.3 verification\_queue.dat.
.2 src.
.3 account.cpp.
.3 account.h.
.3 client.cpp.
.3 client.h.
.3 crypto.cpp.
.3 crypto.h.
.3 database.cpp.
.3 database.h.
.3 init\_database.cpp.
.3 main\_client.cpp.
.3 main\_server.cpp.
.3 server.cpp.
.3 server.h.
.3 view\_database.cpp.
.2 tests.
.3 test\_bank\_system.cpp.
}
\begin{comment}
.
├── CMakeLists.txt
├── Makefile
├── build
│   ├── CMakeCache.txt
│   ├── CMakeFiles
│   ├── CTestTestfile.cmake
│   ├── DartConfiguration.tcl
│   ├── Makefile
│   ├── Testing
│   ├── cmake_install.cmake
│   ├── data
│   └── test_data
├── data
│   ├── accounts.dat
│   ├── accounts.dat.settings
│   └── verification_queue.dat
├── src
│   ├── account.cpp
│   ├── account.h
│   ├── client.cpp
│   ├── client.h
│   ├── crypto.cpp
│   ├── crypto.h
│   ├── database.cpp
│   ├── database.h
│   ├── init_database.cpp
│   ├── main_client.cpp
│   ├── main_server.cpp
│   ├── server.cpp
│   ├── server.h
│   └── view_database.cpp
├── tests
    └── test_bank_system.cpp
\end{comment}

\begin{figure}[H]
\centering
\includegraphics[width=0.99\textwidth]{class-diagram.png}
\caption{Диаграмма классов системы}
\label{fig:class}
\end{figure}

\subsection{Компонентная архитектура}

Программный код можно разделить на следующие ключевые компоненты:
\begin{itemize}
\item \textbf{Client Application} -- клиентское приложение;
\item \textbf{Bank Server} -- многопоточный сервер с системой сессий;
\item \textbf{Database Layer} -- слой работы с данными;
\item \textbf{Security Module} -- модуль безопасности и шифрования.
\end{itemize}

\begin{figure}[H]
\centering
\includegraphics[width=0.9\textwidth]{architecture-diagram.png}
\caption{Архитектурная диаграмма системы}
\label{fig:architecture}
\end{figure}


\subsection{Логика обработки запросов}

\begin{figure}[H]
\centering
\includegraphics[width=0.8\textwidth]{dataflow-diagram.png}
\caption{Логика обработки клиентских запросов}
\label{fig:dataflow}
\end{figure}

\subsection{Архитектурная диаграмма компонент}

\begin{figure}[H]
\centering
\includegraphics[width=\textwidth]{component-diagram.png}
\caption{Диаграмма компонент системы}
\label{fig:component}
\end{figure}

\subsection{Диаграмма последовательности операций}

\begin{figure}[H]
\centering
\includegraphics[width=0.95\textwidth]{sequence-diagram.png}
\caption{Последовательность выполнения операции перевода средств}
\label{fig:sequence}
\end{figure}


\newpage
\section{Пользовательская документация}

\subsection*{Инициализация системы}
\begin{lstlisting}[language=bash]
# Компиляция и настройка
make setup
# Запуск сервера
make run_server
# Запуск клиента (в другом терминале)
make run_client
\end{lstlisting}

\subsection*{Тестовые учетные записи}
\begin{tcolorbox}
\begin{verbatim}
Обычный клиент:         ACC1001 / password123
Обычный клиент:         ACC1002 / qwerty456  
Неверифицированный:     ACC1003 / test789
Сотрудник безопасности: SUPER001 / superpass123
\end{verbatim}
\end{tcolorbox}

\subsection{Основные команды}

\subsubsection{Команды без авторизации:}
\begin{lstlisting}
RATES                                          # Просмотр ставок
REGISTER ФИО 1990-05-15 4510123456 password123 # Регистрация нового пользователя
LOGIN ACC1001 password123                      # Войти в аккаунт
SUPERLOGIN SUPER001 superpass123               # Войти в акккаунт банковского сотрудника
HELP                                           # Справка
\end{lstlisting}

\subsubsection{Команды после авторизации клиента:}
\begin{lstlisting}
ACCOUNTS                # Список счетов
DEPOSIT 1000            # Пополнение на первый счёт 1000 тугриков
WITHDRAW 500            # Снятие с первого счёта 500 тугриков
TRANSFER ACC1002 200    # Перевод пользователю ACC1002 200 тугриков
HISTORY 0               # История операций
CREATE_ACCOUNT 0        # 0=Savings, 1=Checking, 2=Credit, 3=Deposit
INFO                    # Информация о клиенте
LOGOUT                  # Выход из аккаунта
\end{lstlisting}

\subsubsection{Команды сотрудника безопасности:}
\begin{lstlisting}
PENDING_REQUESTS       # Ожидающие операции
PENDING_VERIFICATIONS  # Ожидающие верификации
APPROVE 0              # Одобрить запрос 0
REJECT 1               # Отклонить запрос 1  
VERIFY 0               # Верифицировать клиента 0
SET_RATES 12.0 6.5     # Установить ставки
SETTINGS               # Текущие настройки
\end{lstlisting}

\section{Пример пайплайна работы}

\subsection{Стандартная операция (пополнение счета)}

\begin{figure}[H]
\centering
\includegraphics[width=0.9\textwidth]{standard-operation.png}
\caption{Пайплайн выполнения стандартной операции пополнения счета}
\label{fig:standard_op}
\end{figure}

\subsection{Операция с одобрением (крупный перевод)}

\begin{figure}[H]
\centering
\includegraphics[width=0.9\textwidth]{approval-operation.png}
\caption{Пайплайн операции перевода, требующей одобрения безопасности}
\label{fig:approval_op}
\end{figure}

\subsection{Процесс регистрации нового клиента}

\begin{figure}[H]
\centering
\includegraphics[width=0.9\textwidth]{registration-process.png}
\caption{Процесс регистрации и верификации нового клиента}
\label{fig:registration}
\end{figure}




\newpage
\section{Технические особенности}

\subsection*{Шифрование данных}
\begin{itemize}
\item Алгоритм: XOR + Base64 (для демонстрации, но можно и более продвинутые)
\item Ключ: bank-system-key-2024 (для проверки дешифровальщика)
\item Область применения: файлы базы данных и настроек
\end{itemize}

\subsection*{Сетевое взаимодействие}
\begin{itemize}
\item Протокол: TCP/IP
\item Порт по умолчанию: 8080 (но можно открывать на произвольном)
\item Формат команд: текстовые строки
\item Кодировка: UTF-8 (по умолчанию)
\end{itemize}

\subsection*{Управление памятью}
\begin{itemize}
\item Умные указатели: для автоматического управления памятью
\item RAII и умные указатели: для корректного распределения ресурсов (сокеты, файлы)
\item Контейнеры STL: для хранения данных
\end{itemize}

\subsection*{Обработка ошибок}
\begin{itemize}
\item Исключения: для корректной обработки ошибок
\item Коды возврата: для прослеживания логики работы
\item Логирование: в консоль сервера и системные файлы
\end{itemize}




\newpage
\section{Пайплайн работы над проектом}

\setlist[itemize]{label=--}

\subsection*{Изучение теоретической базы}
\begin{itemize}
    \item Принципы клиент-серверной архитектуры
    \item Сетевые протоколы TCP/IP, сокеты в C++
    \item Многопоточное программирование
    \item Основы криптографии (хеширование, шифрование XOR + Base64)
    \item Паттерны проектирования для банковских систем
    \item Системы аутентификации и авторизации
    \item Принципы безопасного хранения данных
\end{itemize}

\subsection*{Поэтапная разработка системы}

\subsubsection*{Фаза 1: Базовая архитектура (Недели 1-2)}
\textbf{Создание ``костяка'' системы:}
\begin{itemize}
    \item Разработана базовая структура проекта с разделением на модули
    \item Реализованы классы Account, Transaction для представления банковских сущностей
    \item Создана заготовка для системы клиент-серверного взаимодействия
\end{itemize}
\textbf{Основные решения:}
\begin{itemize}
    \item Выбрана трехуровневая архитектура (клиент-сервер-база данных)
    \item Определены основные типы счетов: SAVINGS, CHECKING, CREDIT, DEPOSIT
    \item Разработана структура транзакций с историей операций
\end{itemize}

\subsubsection*{Фаза 2: Сетевое взаимодействие (Недели 3-4)}
\textbf{Реализация клиент-серверной связи:}
\begin{itemize}
    \item Разработан класс BankServer с многопоточным подключением клиентов
    \item Создан класс BankClient для взаимодействия с сервером
    \item Реализован текстовый протокол для команд (в терминале)
\end{itemize}
\textbf{Особенности реализации:}
\begin{itemize}
    \item Многопоточная обработка клиентов через std::thread
    \item Система сессий для управления состоянием клиентов
\end{itemize}

\subsubsection*{Фаза 3: Работа с данными и система аутентификации (Недели 5-7)}
\textbf{Реализация системы хранения:}
\begin{itemize}
    \item Класс Database для управления данными
    \item Шифрование записей в базе данных (создан модуль Crypto для шифрования и хеширования)
    \item Реализована система регистрации и логина
    \item Добавлена двухуровневая аутентификация (клиенты/суперпользователи)
\end{itemize}
\textbf{Особенности реализации:}
\begin{itemize}
    \item Автоматическое сохранение при изменениях
    \item Резервное копирование и восстановление
    \item Формат хранения с поддержкой сохранения истории транзакций
\end{itemize}
\textbf{Реализовано:}
\begin{itemize}
    \item REGISTER -- регистрация новых пользователей
    \item LOGIN -- вход для обычных клиентов
    \item SUPERLOGIN -- вход для сотрудников безопасности
    \item Реализована утилита view\_database для просмотра зашифрованных данных
    \item Создан скрипт инициализации тестовых данных init\_database
    \item Добавлено логирование ключевых событий на сервере
\end{itemize}

\subsubsection*{Фаза 4: Базовые банковские операции (Неделя 8)}
\textbf{Реализация основных функций:}
\begin{itemize}
    \item DEPOSIT / DEPOSIT\_TO -- пополнение счетов
    \item WITHDRAW / WITHDRAW\_FROM -- снятие средств
    \item TRANSFER / TRANSFER\_FROM -- переводы между счетами
    \item CREATE\_ACCOUNT -- создание новых счетов
\end{itemize}
\textbf{Особенности реализации:}
\begin{itemize}
    \item Автоматическое ведение истории транзакций
    \item Поддержка кредитных лимитов для счетов
\end{itemize}

\subsubsection*{Фаза 5: Система одобрения операций (Недели 9-10)}
\textbf{Реализация безопасности для крупных операций:}
\begin{itemize}
    \item Механизм очередей запросов на одобрение
    \item Команды для суперпользователей: APPROVE, REJECT, PENDING\_REQUESTS
    \item Автоматическое определение крупных операций по пороговому значению
\end{itemize}
\textbf{Особенности реализации:}
\begin{itemize}
    \item Операции выше порога требуют одобрения безопасности
    \item Система ожидания ответа с таймаутом
    \item Уведомления клиентов о статусе операций
\end{itemize}

\subsubsection*{Фаза 6: Верификация пользователей (Недели 9-10)}
\textbf{Система проверки клиентов:}
\begin{itemize}
    \item Статусы пользователей: PENDING\_VERIFICATION, VERIFIED, BLOCKED (в процессе)
    \item Очередь запросов на верификацию
    \item Команды VERIFY, PENDING\_VERIFICATIONS для суперпользователей
\end{itemize}
\textbf{Для неверефицированных пользователей установлены:}
\begin{itemize}
    \item Лимиты на размер операций
    \item Запрет на создание кредитных и депозитных счетов
\end{itemize}
\emph{Для верифицированных пользователей остаётся доступен весь ранее описанный функционал.}

\subsubsection*{Фаза 7: Процентные ставки и настройки (Неделя 11, не закончено)}
\textbf{Управление банковскими параметрами:}
\begin{itemize}
    \item Система хранения настроек банка (лимиты, ставки)
    \item Команды SET\_RATES, SETTINGS для управления ставками
    \item Динамическое применение изменений
\end{itemize}

\subsubsection*{Фаза 8: Тестирование системы (Недели 12-13)}

\subsubsection*{Модульное тестирование}
\begin{itemize}
    \item Создан комплексный тестовый класс BankSystemTest
    \item Тестирование основных сценариев: регистрация, логин, операции
    \item Проверка обработки ошибок и граничных случаев
\end{itemize}

\subsubsection*{Интеграционное тестирование}
\begin{itemize}
    \item Тестирование взаимодействия клиент-сервер
    \item Проверка работы системы одобрения операций
    \item Тестирование восстановления после сбоев
\end{itemize}

\subsubsection*{Нагрузочное тестирование}
\begin{itemize}
    \item Проверка многопользовательской работы
    \item Тестирование обработки параллельных запросов
\end{itemize}

\subsubsection*{Инструменты тестирования}
\begin{itemize}
    \item Google Test framework для unit-тестов
    \item Автоматизированные сценарии для интеграционного тестирования
    \item Ручное тестирование пользовательских сценариев
\end{itemize}

\subsubsection*{Документирование}
\begin{itemize}
    \item Создана документация на русском языке
    \item Описаны функциональные и системные требования
    \item Разработаны архитектурные диаграммы и схемы работы
    \item Создано пользовательское руководство с примерами
\end{itemize}

\subsubsection*{Оптимизация и рефакторинг}
\begin{itemize}
    \item Улучшена обработка ошибок
    \item Оптимизирована работа с памятью
    \item Улучшена производительность сетевого взаимодействия
\end{itemize}

\subsection*{Итог работы}
Проект развивался по принципу ``снизу вверх'' -- от базовых структур данных к сложной логике взаимодействия клиента, сервера и суперпользователя. Каждый этап включал проектирование, реализацию, а также обязательное тестирование и интеграцию (что облегчалось, так как старались придерживаться модульности приложения по SOLID). Такой подход позволил создать стабильную, безопасную и \underline{расширяемую} банковскую систему.






\newpage
\section{Заключение}

Secure Bank System представляет собой комплексное решение для безопасного банковского обслуживания с модульной архитектурой и многоуровневой системой безопасности. Система демонстрирует лучшие практики разработки финансовых приложений и может служить основой для более сложных банковских систем.

\begin{thebibliography}{9}
\bibitem{cpp17} ISO/IEC 14882:2017 - Programming Language C++
\bibitem{securecoding} Seacord, R. C. (2013). \emph{Secure Coding in C and C++}
\bibitem{networkprogramming} Donahoo, M. J., \& Calvert, K. L. (2001). \emph{TCP/IP Sockets in C: Practical Guide for Programmers}
\bibitem{yandex_raii_2023}
RAII и умные указатели // Справочник по C++ Яндекс [Электронный ресурс]. URL: https://education.yandex.ru/handbook/cpp/article/raii-and-smart-pointers.
\bibitem{deepwiki_base64_xor}
StivenHacker. Base64 and XOR Encryption // DeepWiki [Электронный ресурс]. URL: https://deepwiki.com/stivenhacker/GhostStrike/3.1-base64-and-xor-encryption.
\end{thebibliography}

\end{document}

